% Options for packages loaded elsewhere
\PassOptionsToPackage{unicode}{hyperref}
\PassOptionsToPackage{hyphens}{url}
%
\documentclass[
]{article}
\usepackage{amsmath,amssymb}
\usepackage{iftex}
\ifPDFTeX
  \usepackage[T1]{fontenc}
  \usepackage[utf8]{inputenc}
  \usepackage{textcomp} % provide euro and other symbols
\else % if luatex or xetex
  \usepackage{unicode-math} % this also loads fontspec
  \defaultfontfeatures{Scale=MatchLowercase}
  \defaultfontfeatures[\rmfamily]{Ligatures=TeX,Scale=1}
\fi
\usepackage{lmodern}
\ifPDFTeX\else
  % xetex/luatex font selection
\fi
% Use upquote if available, for straight quotes in verbatim environments
\IfFileExists{upquote.sty}{\usepackage{upquote}}{}
\IfFileExists{microtype.sty}{% use microtype if available
  \usepackage[]{microtype}
  \UseMicrotypeSet[protrusion]{basicmath} % disable protrusion for tt fonts
}{}
\makeatletter
\@ifundefined{KOMAClassName}{% if non-KOMA class
  \IfFileExists{parskip.sty}{%
    \usepackage{parskip}
  }{% else
    \setlength{\parindent}{0pt}
    \setlength{\parskip}{6pt plus 2pt minus 1pt}}
}{% if KOMA class
  \KOMAoptions{parskip=half}}
\makeatother
\usepackage{xcolor}
\usepackage[margin=1in]{geometry}
\usepackage{color}
\usepackage{fancyvrb}
\newcommand{\VerbBar}{|}
\newcommand{\VERB}{\Verb[commandchars=\\\{\}]}
\DefineVerbatimEnvironment{Highlighting}{Verbatim}{commandchars=\\\{\}}
% Add ',fontsize=\small' for more characters per line
\usepackage{framed}
\definecolor{shadecolor}{RGB}{248,248,248}
\newenvironment{Shaded}{\begin{snugshade}}{\end{snugshade}}
\newcommand{\AlertTok}[1]{\textcolor[rgb]{0.94,0.16,0.16}{#1}}
\newcommand{\AnnotationTok}[1]{\textcolor[rgb]{0.56,0.35,0.01}{\textbf{\textit{#1}}}}
\newcommand{\AttributeTok}[1]{\textcolor[rgb]{0.13,0.29,0.53}{#1}}
\newcommand{\BaseNTok}[1]{\textcolor[rgb]{0.00,0.00,0.81}{#1}}
\newcommand{\BuiltInTok}[1]{#1}
\newcommand{\CharTok}[1]{\textcolor[rgb]{0.31,0.60,0.02}{#1}}
\newcommand{\CommentTok}[1]{\textcolor[rgb]{0.56,0.35,0.01}{\textit{#1}}}
\newcommand{\CommentVarTok}[1]{\textcolor[rgb]{0.56,0.35,0.01}{\textbf{\textit{#1}}}}
\newcommand{\ConstantTok}[1]{\textcolor[rgb]{0.56,0.35,0.01}{#1}}
\newcommand{\ControlFlowTok}[1]{\textcolor[rgb]{0.13,0.29,0.53}{\textbf{#1}}}
\newcommand{\DataTypeTok}[1]{\textcolor[rgb]{0.13,0.29,0.53}{#1}}
\newcommand{\DecValTok}[1]{\textcolor[rgb]{0.00,0.00,0.81}{#1}}
\newcommand{\DocumentationTok}[1]{\textcolor[rgb]{0.56,0.35,0.01}{\textbf{\textit{#1}}}}
\newcommand{\ErrorTok}[1]{\textcolor[rgb]{0.64,0.00,0.00}{\textbf{#1}}}
\newcommand{\ExtensionTok}[1]{#1}
\newcommand{\FloatTok}[1]{\textcolor[rgb]{0.00,0.00,0.81}{#1}}
\newcommand{\FunctionTok}[1]{\textcolor[rgb]{0.13,0.29,0.53}{\textbf{#1}}}
\newcommand{\ImportTok}[1]{#1}
\newcommand{\InformationTok}[1]{\textcolor[rgb]{0.56,0.35,0.01}{\textbf{\textit{#1}}}}
\newcommand{\KeywordTok}[1]{\textcolor[rgb]{0.13,0.29,0.53}{\textbf{#1}}}
\newcommand{\NormalTok}[1]{#1}
\newcommand{\OperatorTok}[1]{\textcolor[rgb]{0.81,0.36,0.00}{\textbf{#1}}}
\newcommand{\OtherTok}[1]{\textcolor[rgb]{0.56,0.35,0.01}{#1}}
\newcommand{\PreprocessorTok}[1]{\textcolor[rgb]{0.56,0.35,0.01}{\textit{#1}}}
\newcommand{\RegionMarkerTok}[1]{#1}
\newcommand{\SpecialCharTok}[1]{\textcolor[rgb]{0.81,0.36,0.00}{\textbf{#1}}}
\newcommand{\SpecialStringTok}[1]{\textcolor[rgb]{0.31,0.60,0.02}{#1}}
\newcommand{\StringTok}[1]{\textcolor[rgb]{0.31,0.60,0.02}{#1}}
\newcommand{\VariableTok}[1]{\textcolor[rgb]{0.00,0.00,0.00}{#1}}
\newcommand{\VerbatimStringTok}[1]{\textcolor[rgb]{0.31,0.60,0.02}{#1}}
\newcommand{\WarningTok}[1]{\textcolor[rgb]{0.56,0.35,0.01}{\textbf{\textit{#1}}}}
\usepackage{graphicx}
\makeatletter
\def\maxwidth{\ifdim\Gin@nat@width>\linewidth\linewidth\else\Gin@nat@width\fi}
\def\maxheight{\ifdim\Gin@nat@height>\textheight\textheight\else\Gin@nat@height\fi}
\makeatother
% Scale images if necessary, so that they will not overflow the page
% margins by default, and it is still possible to overwrite the defaults
% using explicit options in \includegraphics[width, height, ...]{}
\setkeys{Gin}{width=\maxwidth,height=\maxheight,keepaspectratio}
% Set default figure placement to htbp
\makeatletter
\def\fps@figure{htbp}
\makeatother
\usepackage{svg}
\setlength{\emergencystretch}{3em} % prevent overfull lines
\providecommand{\tightlist}{%
  \setlength{\itemsep}{0pt}\setlength{\parskip}{0pt}}
\setcounter{secnumdepth}{-\maxdimen} % remove section numbering
\ifLuaTeX
  \usepackage{selnolig}  % disable illegal ligatures
\fi
\usepackage{bookmark}
\IfFileExists{xurl.sty}{\usepackage{xurl}}{} % add URL line breaks if available
\urlstyle{same}
\hypersetup{
  pdftitle={How does a pendulum's length affect its time period?},
  pdfauthor={Heather Wang, Qihan Liu, Runxi Yu, Jihoo(Vincent) Bae},
  hidelinks,
  pdfcreator={LaTeX via pandoc}}

\title{How does a pendulum's length affect its time period?}
\author{Heather Wang, Qihan Liu, Runxi Yu, Jihoo(Vincent) Bae}
\date{}

\begin{document}
\maketitle

\subsection{Introduction}\label{introduction}

In this experiment, we investigate whether changing the length of a
pendulum's string causes changes in its period (the time it takes to
complete one full swing back and forth).

\begin{figure}
\centering
\includesvg{Simple_gravity_pendulum.svg}
\caption{Simple gravity pendulum}
\end{figure}

Theory suggests that for sufficiently small amounts angles, the pendulum
follows \(T = 2\pi\sqrt{\frac{L}{g}}\) where \(T\) is the period, \(L\)
is the length, and \(g\) is the acceleration due to gravity. Therefore,
the theory suggests that changes in the string's length do affect the
time period.

Therefore, our hypotheses shall be: * \textbf{Null hypothesis:} Changes
in the string's length do not affect the time period. *
\textbf{Alternate hypothesis:} Changes in the string's length do affect
the time period.

\subsection{Treatment}\label{treatment}

The treatment in this experiment is the pendulum length, for which we
suggested \(10\,\mathrm{cm}\), \(15\,\mathrm{cm}\), \(20\,\mathrm{cm}\),
and \(25\,\mathrm{cm}\).

Random assignment is not applicable in this context because the
experiment is designed as a deterministic system in a controlled
environment; all extraneous variables, such as the mass of the bulb, the
gravitational field strength of the earth, were held constant as far as
possible. Each length was tested in four trials to reduce measurement
errors.

There are some limitations on generalizability. First, although the
theoretical formula assumes small-angle oscillations, the initial
release angles in this experiment were relatively large. This introduces
minor discrepancies due to nonlinear effects at larger angles.
Additionally, the results may only generalize to pendulums with similar
construction materials and in similar air conditions (e.g., negligible
air resistance).

\subsection{Subjects}\label{subjects}

The subjects are the pendulums constructed by tying two 1 CNY coins to a
length of string. The same type and number of coins were used throughout
the experiment. A piece of string was suspended from a fixed support,
and the coins were attached to the end of the string to act as the
pendulum bob. The pendulum was released from a fixed angle, and the time
it took to complete one period was recorded using a stopwatch.

2 coins were chosen as they are readily available, of a fixed mass, and
provide enough mass for the pendulum to swing. The subjects are ideal
for the study since the pendulum system is a physical object rather than
a social or biological population. They directly represent the physical
system being investigated and allow for controlled manipulation of the
treatment variable while ensuring other factors are constant. There
might be limitations in generalizing the results to all pendulums. The
findings will directly apply to pendulums made with similar materials
and in similar settings, but may not be generalizable to pendulums with
different bobs, materials, or conditions. That said, while the exact
timings and results from this experiment is not applicable to all
pendulums, the trend and relationship between pendulum length and period
are generalizable.

\subsection{Analysis}\label{analysis}

\subsubsection{Data overview}\label{data-overview}

\begin{Shaded}
\begin{Highlighting}[]
\NormalTok{data }\OtherTok{\textless{}{-}} \FunctionTok{read.csv}\NormalTok{(}\StringTok{"data.csv"}\NormalTok{)}
\NormalTok{data}
\end{Highlighting}
\end{Shaded}

\begin{verbatim}
##    length_cm time_10_periods
## 1         25           11.01
## 2         25           10.93
## 3         25           11.01
## 4         25           10.94
## 5         20           10.60
## 6         20           10.26
## 7         20           10.16
## 8         20           10.28
## 9         15            9.28
## 10        15            9.18
## 11        15            9.41
## 12        15            9.61
## 13        10            8.40
## 14        10            8.35
## 15        10            8.70
## 16        10            8.76
\end{verbatim}

\begin{Shaded}
\begin{Highlighting}[]
\FunctionTok{summary}\NormalTok{(data)}
\end{Highlighting}
\end{Shaded}

\begin{verbatim}
##    length_cm     time_10_periods 
##  Min.   :10.00   Min.   : 8.350  
##  1st Qu.:13.75   1st Qu.: 9.075  
##  Median :17.50   Median : 9.885  
##  Mean   :17.50   Mean   : 9.805  
##  3rd Qu.:21.25   3rd Qu.:10.682  
##  Max.   :25.00   Max.   :11.010
\end{verbatim}

Treating the length as discrete groups, let us estimate the mean and
standard deviation of each group.

\begin{Shaded}
\begin{Highlighting}[]
\FunctionTok{aggregate}\NormalTok{(time\_10\_periods }\SpecialCharTok{\textasciitilde{}}\NormalTok{ length\_cm, }\AttributeTok{data =}\NormalTok{ data, }\ControlFlowTok{function}\NormalTok{(x) }\FunctionTok{c}\NormalTok{(}\AttributeTok{mean =} \FunctionTok{mean}\NormalTok{(x), }\AttributeTok{sd =} \FunctionTok{sd}\NormalTok{(x), }\AttributeTok{n =} \FunctionTok{length}\NormalTok{(x)))}
\end{Highlighting}
\end{Shaded}

\begin{verbatim}
##   length_cm time_10_periods.mean time_10_periods.sd time_10_periods.n
## 1        10           8.55250000         0.20742469        4.00000000
## 2        15           9.37000000         0.18565200        4.00000000
## 3        20          10.32500000         0.19070046        4.00000000
## 4        25          10.97250000         0.04349329        4.00000000
\end{verbatim}

This allows us to see that there is a significant Difference in Means
betwen groups, of approximately 0.6 between neighbouring groups, where
the standard deviation is roughly around 0.1 to 0.2.

In order to gain a general sense of the data we have obtained, let us
plot the data. A box plot is used, to present the information aggregated
by group above.

\begin{Shaded}
\begin{Highlighting}[]
\FunctionTok{boxplot}\NormalTok{(time\_10\_periods }\SpecialCharTok{\textasciitilde{}}\NormalTok{ length\_cm, }\AttributeTok{data =}\NormalTok{ data,}
    \AttributeTok{main =} \StringTok{"Effect of Length on Time"}\NormalTok{,}
    \AttributeTok{xlab =} \StringTok{"Length (cm)"}\NormalTok{,}
    \AttributeTok{ylab =} \StringTok{"Time (10 periods)"}\NormalTok{)}
\end{Highlighting}
\end{Shaded}

\includegraphics{test_files/figure-latex/unnamed-chunk-3-1.pdf}

\subsubsection{Linear regression}\label{linear-regression}

\begin{Shaded}
\begin{Highlighting}[]
\NormalTok{model }\OtherTok{\textless{}{-}} \FunctionTok{lm}\NormalTok{(time\_10\_periods }\SpecialCharTok{\textasciitilde{}}\NormalTok{ length\_cm, }\AttributeTok{data =}\NormalTok{ data)}
\FunctionTok{summary}\NormalTok{(model)}
\end{Highlighting}
\end{Shaded}

\begin{verbatim}
## 
## Call:
## lm(formula = time_10_periods ~ length_cm, data = data)
## 
## Residuals:
##      Min       1Q   Median       3Q      Max 
## -0.22275 -0.10900 -0.02725  0.08000  0.38425 
## 
## Coefficients:
##             Estimate Std. Error t value Pr(>|t|)    
## (Intercept) 6.929750   0.141582   48.95  < 2e-16 ***
## length_cm   0.164300   0.007707   21.32 4.52e-12 ***
## ---
## Signif. codes:  0 '***' 0.001 '**' 0.01 '*' 0.05 '.' 0.1 ' ' 1
## 
## Residual standard error: 0.1723 on 14 degrees of freedom
## Multiple R-squared:  0.9701, Adjusted R-squared:  0.968 
## F-statistic: 454.5 on 1 and 14 DF,  p-value: 4.516e-12
\end{verbatim}

We shall run various diagnostic plots of this linear regression model to
confirm whether it is truly a linear regression:

\begin{Shaded}
\begin{Highlighting}[]
\FunctionTok{plot}\NormalTok{(model)}
\end{Highlighting}
\end{Shaded}

\includegraphics{test_files/figure-latex/unnamed-chunk-5-1.pdf}
\includegraphics{test_files/figure-latex/unnamed-chunk-5-2.pdf}
\includegraphics{test_files/figure-latex/unnamed-chunk-5-3.pdf}
\includegraphics{test_files/figure-latex/unnamed-chunk-5-4.pdf}

There are insufficient sample points for the diagnostic plots to be
accurate, but they seem to be acceptable for this sample size. It is
however somewhat unclear what causes the enlarged residues for large
theoretical quantities.

Given that a linear regression is an appropriate estimate as suggested
by the diagnostic tests (even though the relationship is theoretically
supposed to be that of a square root), its \(p\)-value
\(4.52 \times 10^{-12}\) suggests that it is extremely unlikely to
obtain our result (or more extreme results) given that the null
hypothesis is true, therefore suggesting with high confidence that the
null hypothesis is false and that the alternate hypothesis is true.
Therefore: changes in the string's length do affect the time period.

\subsection{T-tests}\label{t-tests}

Another way to confirm the statistical significance of the difference in
means is via Welch's \(t\)-test, not assuming equal variances.

\begin{Shaded}
\begin{Highlighting}[]
\NormalTok{group\_10 }\OtherTok{\textless{}{-}}\NormalTok{ data}\SpecialCharTok{$}\NormalTok{time\_10\_periods[data}\SpecialCharTok{$}\NormalTok{length\_cm }\SpecialCharTok{==} \DecValTok{10}\NormalTok{]}
\NormalTok{group\_15 }\OtherTok{\textless{}{-}}\NormalTok{ data}\SpecialCharTok{$}\NormalTok{time\_10\_periods[data}\SpecialCharTok{$}\NormalTok{length\_cm }\SpecialCharTok{==} \DecValTok{15}\NormalTok{]}
\NormalTok{group\_20 }\OtherTok{\textless{}{-}}\NormalTok{ data}\SpecialCharTok{$}\NormalTok{time\_10\_periods[data}\SpecialCharTok{$}\NormalTok{length\_cm }\SpecialCharTok{==} \DecValTok{20}\NormalTok{]}
\NormalTok{group\_25 }\OtherTok{\textless{}{-}}\NormalTok{ data}\SpecialCharTok{$}\NormalTok{time\_10\_periods[data}\SpecialCharTok{$}\NormalTok{length\_cm }\SpecialCharTok{==} \DecValTok{25}\NormalTok{]}
\end{Highlighting}
\end{Shaded}

\begin{Shaded}
\begin{Highlighting}[]
\FunctionTok{print}\NormalTok{(}\FunctionTok{t.test}\NormalTok{(group\_10, group\_15, }\AttributeTok{var.equal =} \ConstantTok{FALSE}\NormalTok{))}
\end{Highlighting}
\end{Shaded}

\begin{verbatim}
## 
##  Welch Two Sample t-test
## 
## data:  group_10 and group_15
## t = -5.8734, df = 5.9277, p-value = 0.001126
## alternative hypothesis: true difference in means is not equal to 0
## 95 percent confidence interval:
##  -1.1590869 -0.4759131
## sample estimates:
## mean of x mean of y 
##    8.5525    9.3700
\end{verbatim}

\begin{Shaded}
\begin{Highlighting}[]
\FunctionTok{print}\NormalTok{(}\FunctionTok{t.test}\NormalTok{(group\_15, group\_20, }\AttributeTok{var.equal =} \ConstantTok{FALSE}\NormalTok{))}
\end{Highlighting}
\end{Shaded}

\begin{verbatim}
## 
##  Welch Two Sample t-test
## 
## data:  group_15 and group_20
## t = -7.1765, df = 5.9957, p-value = 0.000371
## alternative hypothesis: true difference in means is not equal to 0
## 95 percent confidence interval:
##  -1.2806739 -0.6293261
## sample estimates:
## mean of x mean of y 
##     9.370    10.325
\end{verbatim}

\begin{Shaded}
\begin{Highlighting}[]
\FunctionTok{print}\NormalTok{(}\FunctionTok{t.test}\NormalTok{(group\_20, group\_25, }\AttributeTok{var.equal =} \ConstantTok{FALSE}\NormalTok{))}
\end{Highlighting}
\end{Shaded}

\begin{verbatim}
## 
##  Welch Two Sample t-test
## 
## data:  group_20 and group_25
## t = -6.6207, df = 3.3113, p-value = 0.005125
## alternative hypothesis: true difference in means is not equal to 0
## 95 percent confidence interval:
##  -0.9428253 -0.3521747
## sample estimates:
## mean of x mean of y 
##   10.3250   10.9725
\end{verbatim}

Again, the \(p\)-values are extremely small, suggesting that it is
extremely unlikely to obtain our result (or more extreme results) given
that the null hypothesis is true, therefore suggesting with high
confidence that the null hypothesis is false and that the alternate
hypothesis is true. Therefore: changes in the string's length do affect
the time period.

\subsection{Estimated effects}\label{estimated-effects}

Are effects estimated correctly? Are they described correctly? Is their
substantive significance discussed, ideally with a relevant benchmark?

\subsection{Statistical Significance}\label{statistical-significance}

Are p-values presented and discussed correctly? A full score would
discuss why the p-value is obtained, i.e.~what about the study led to
the p-values estimated.

\emph{I think this is done}

\subsection{Informal bibliography}\label{informal-bibliography}

I'm not really sure how to get BibLaTeX working with R Markdown yet, so
here's just an itemized list of references without any particular
bibliography format.

\begin{itemize}
\tightlist
\item
  Simple Gravity pendulum, by Chetvorno, public domain,
  \url{https://commons.wikimedia.org/w/index.php?curid=5276335}
\end{itemize}

\end{document}
